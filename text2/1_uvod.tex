% (a) Úvod charakterizující kontext zadání.
\begin{introduction}
Když jsem se v květnu roku 2011 začal zabývat diplomovou prací, neměl jsem v tom, co budu dělat, vůbec jasno. Jedním z mála tehdejších cílů bylo vytvoření aplikace, ze které bude mít užitek co největší počet lidí. V rámci bakalářské práce jsem se zabýval doménou navigace studentů po Fakultě elektrotechnické \glsname{CVUT} \cite{Bakalarka}, proto jsem, mezi jinými tématy, zvažoval i pokračování v obdobné doméně na Fakultě informačních technologií -- jedná se o oblast, ve které lze studentům přinést mnoho užitečného.

Nejprve jsem zkontaktoval pana docenta Vitvara, kterého téma oslovilo a začali jsme společně tvořit zadání diplomové práce. Dohodli jsme se na struktuře práce, použitých technologiích a některých dalších detailech. Mezi požadavky se objevila tvorba mobilní aplikace pro přístup k informacím, těmi se pan docent nezabývá, proto mi doporučil další pokračování práce pod panem inženýrem Havrylukem, jehož jsou primární doménou. Této nabídky jsem využil a dále pokračoval pod vedením pana inženýra Havryluka.

Shodná doména bakalářské a diplomové práce se potvrdila být prospěšnou -- ačkoliv jsem z bakalářské práce v diplomové mimo získaného přehledu o doméně nakonec nic jiného nevyužil, právě ten se ukázal být jako velmi přínosný -- od začátku jsem měl přehled o problematických místech a těch, které tehdy nebylo možné realizovat, a mohl se na ně zaměřit a vyřešit tentokrát podstatně lépe. Bakalářská práce se zabývala pouze navigací do určitého bodu, diplomová jde podstatně více do hloubky i do šířky a přináší celého průvodce postaveného na solidních základech.
\end{introduction}
